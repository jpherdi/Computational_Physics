\section*{Aufgabe 2}

\begin{enumerate}[label=\alph*)]

    \item Diese Aufgabe haben wir zeitlich diese Woche nicht komplett geschafft. Wir haben die Tridiagonalisierung durch den Lanczos-Algorithmus implementiert wie in der Vorlesung vorgegeben. Leider funktioniert dieser noch nicht wie er soll und ihr braucht euch das nicht anzuschauen.

    \item    
    Die Matrix $H$ ist eine Tridiagonalmatrix wobei auf den beiden Nebendiagonalreihen jeweils $-t$ steht und nur das Diagonalelement $H(N/2, N/2)$ ist $\epsilon$. Für Dimension $N=6$ ist diese Matrix
    \begin{equation*}
        \begin{pmatrix}
            0& -t& 0& 0& 0& 0 \\
            -t&  0& -t& 0& 0& 0 \\
            0&  -t& \eta& -t& 0& 0 \\
            0&  0& -t& 0& -t& 0 \\
            0&  0& 0& -t& 0& -t \\
            0&  0& 0& 0& -t& 0 \\
        \end{pmatrix}.
    \end{equation*}
    

\end{enumerate}
