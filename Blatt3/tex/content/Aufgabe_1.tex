\section*{Aufgabe 1}

\begin{enumerate}[label=\alph*)]

\item Gegeben ist eine freie lineare Federkette mit den Punktmassen $m_i$, und zwischen den Massen die Federkonstanten $k_j$ und die Federruhelänge $l_j$ mit $i \in \{1, \dots, N\}$ und $j \in \{1, \dots, N-1\}$.
Hier wirken jeweils die Nachbarn mit dem Index $i-1, i+1$ auf die Punktmasse $m_i$.
Die Bewegungsgleichung ergibt sich mit den 
Relativkoordinaten $\eta_i$ zu $\vec \eta = \left(0, 0+l_1, 0+ l_1 + l_2, \dots\right)^T$ also mit $l_0 = 0$ lässt sich $\eta_i$ als
\begin{equation*}
    \eta_i = \sum_j^{i} l_j.
\end{equation*}
Nach Newton gilt $m_i \ddot{\eta_i} = - \frac{\partial}{\partial \eta_i} V(\eta_1, \eta_2, \dots, \eta_N).$
Dadurch sich die Matrixgleichung 
\begin{equation*}
    M \ddot{\vec{\eta}} + K \vec{\eta} = 0.    
\end{equation*}

Dabei ist die $N \times N$-Matrix 
\begin{equation*}
    M = \begin{pmatrix*}
        m_1 &0   &0     &0 \\
        0   &m_2 &0     &0 \\
        0   &0   &\dots &0 \\
        0   &0   &0     &m_N \\
    \end{pmatrix*},
\end{equation*}
mit den Massen auf der Diagonalen. 
Die Matrix $K$ ist eine Tridiagonalmatrix mit 
\begin{equation*}
    K = \begin{pmatrix*}
        k_1 & -k_1   &0     &0 \\
        -k_1   &k_1+k_2 &-k_2     &0 \\
        0   &\dots   &\dots &\dots \\
        0   &0   &-k_{N-1}     &k_{N-1} \\
    \end{pmatrix*}
\end{equation*}
definiert. 
Die Lösung der Bewegungsgleichung lässt sich über den Ansatz
\begin{equation*}
    \eta_i = A_i \cdot \exp(i \omega t)
\end{equation*}
lösen. Dadurch ergibt sich die Matrix 
\begin{equation*}
    (K\cdot M^{-1} - \omega^2 \times \mathbb{1}) \cdot \vec \eta = \vec 0
\end{equation*}
Durch Lösen des Eigenwertproblems ergeben sich die Quadrate der Eigenfrequenzen $\omega^2$ und die Bewegungsgleichungen als die Eigenvektoren $\vec \eta$. 

In dieser Aufgabe sind wir an den Eigenwerten $\omega_i$ interessiert. Wir implementieren die Matrizen $K$ und $M^{-1}$ und multiplizieren diese miteinander. Anschließend bestimmen wir mit der \texttt{Eigen}-Funktion \texttt{eigenvalues} die Eigenwerte. Durch das Wurzelziehen der Eigenwerte lassen sich im Anschluss die Eigenfrequenzen $\omega$ bestimmen.

\item In diesem Aufgabenteil waren die Massen, Federkonstanten und Längen definiert mit 
\begin{align*}
    m_i &= i,\\
    k_j &= N-j,\\
    l_j &= |5-j|+1.\\
\end{align*}
Alle physikalischen Größen sind einheitenlos. 
Für ein System mit $N=10$ ergibt sich 
\begin{equation*}
    \vec \omega = \begin{pmatrix*}
        3.95439 \\
        2.62228 \\
        1.95382 \\
        1.51083 \\
        1.17586 \\
       0.901365 \\
       0.663369 \\
       \num{4.18e-09} \\
        0.44762 \\
       0.243446 \\
    \end{pmatrix*}.
\end{equation*}
Einer der Einträge ist von der Ordnung $\mathcal{O}(\num{e-9})$. Dies könnte entweder eine Null sein oder entspricht einem numerischen Problem.
\end{enumerate}

