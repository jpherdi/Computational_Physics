\section*{Aufgabe 1}

In dieser Aufgabe sollen verschiedene Formeln zur Approximation von Integralen implementiert werden.
Die Schrittweite soll so lange halbiert werden, bis die relative Änderung des Ergebnisses kleiner als $10^{-4}$ ist.
Eine Halbierung der Schrittweite $h$ entspricht einer Verdopplung der Intervallanzahl $n$, da $h = \frac{b-a}{n}$.
\newline
Die zu bestimmenden Integrale sind
\begin{align*}
    I_1 &= \int_{1}^{100} dx \frac{e^{-x}}{x} \\
    I_2 &= \int_{0}^{1} dx sin \left( \frac{1}{x} \right.
\end{align*}


\subsection*{a) Trapezregel}
Damit die relative Abweichung der Ergebnisse kleiner als $10^{-4}$ wird, sind mit der Trapezregel für das erste Integral $I_1$ $i = \dots$ Halbierungen nötig. Die Schrittweite beträgt dann $h = \dots$.
\newline
Bei dem zweiten Integral $I_2$ muss $i = \dots$ mal halbiert werden und die Schrittweite beträgt $h = \dots$.

\subsection*{b) Mittelpunktsregel}
Für das erste Integral beträgt die Zahl der Halbierungen $i = \dots$ und die Schrittweite $h = \dots$.
\newline
Bei dem zweiten Integral sind mit der Mittelpunktsregel $i = \dots$ Halbierungen nötig und eine Schrittweite von $h = \dots$ resultiert.

\subsection*{c) Simpsonregel}
Bei Verwendung der Simpsonregel muss für das erste Integral die Schrittweite $i = \dots$ mal halbiert werden, damit die Abweichung klein genug ist. Die Schrittweite beträgt schließlich $h = \dots$.
\newline
Für das zweite Integral sind die Werte $i = \dots$ und $h = \dots$.

\subsection*{Fazit}