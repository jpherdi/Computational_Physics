\section*{Aufgabe 2}

\begin{enumerate}[label=\alph*)]
    \item Ein zufälliges Gleichungssystem $Mx = b$ soll mit den Funktionen \texttt{Random} Funktionen von \texttt{Eigen} erstellt werden. 
    Eine Variable $N$ gibt die Dimension der $N \cross N$ Matrix und die Länge $N$ des Vektors an.
    Im ersten Schritt werden dann die Zufallsgrößen erstellt, im zweiten wird die LU-Zerlegung mit \texttt{Eigen} durchgeführt und am Ende wird das System mit der \texttt{solve}-Funktion gelöst.
    \item Das ganze wird nun für $N = \{1, 2, \dots, 1000\}$ mit einem for-loop durchgeführt. Durch das gegebene \texttt{profiler} Modul, kann die Zeit für die einzelnen Prozesse gespeichert und gestoppt werden. 
    Diese werden doppellogarithmisch abgespeichert und sind in Abb. \ref{fig:times} dargestellt.

    \begin{figure}
        \centering
        \includegraphics[width=0.8\textwidth]{\path/output/times.pdf}
        \caption{Die Zeiten für das erzeugen der Zufallsdaten, der LU-Zerlegung und der anschließenden Lösung des LGS sind in Abhängigkeit der Dimension $N$ doppellogarithmisch dargestellt.}
        \label{fig:times}
    \end{figure}
    \item Die Zeit scheint durch die LU-Zerlegung dominiert zu werden. Diese geht offenbar mit $\mathcal{O}(N^3)$, da mit einer Größenordnung in $N$ die Zeitgrößenordnung um einen Faktor \num{3} zunimmt. Das erzeugen der zufälligen Vektoren und Matrizen geht mit $\mathcal{O}(N^2)$ und das Lösen geht mit $\mathcal{O}(N)$. Das ist alles wie erwartet.
    Bei $N = \num{1000000}$ würde das Programm also in etwa \num{9} Größenordnungen länger dauern als bei $N=\num{1000}$, was einer Zeit von \SI{1e8}{\second} entspricht. 
    Die Optimierung würde sich offenbar am meisten bzgl. der LU-Zerlegung auszahlen. Solange diese mit $\mathcal{O}(N^3)$ läuft, sind die anderen beiden Teile des Programms für die Laufzeit zu vernachlässigen.
    \item 
\end{enumerate}