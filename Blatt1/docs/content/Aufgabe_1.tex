\section*{Aufgabe 1}

\begin{enumerate}[label=\alph*)]

\item Das Kristallsystem ist ein Hexagonalgitter.

\item Eine Fehlstelle befindet sich bei $\vec x = (2,0,2)^T$. Die Koordinaten in der Basis $\{\vec a_1, \vec a_2, \vec a_3\}$ sollen ohne Matrix Inversion gelöst werden.
Das Problem lässt sich als Lineares Gleichungssystem darstellen in der Form $\mathbf{A} \cdot \vec x' = \vec x$.

\item 
\item new
\item Wären die primitiven Gittervektoren paarweise orthogonal, würde sich das Problem auf eine Diagonalmatrix vereinfachen und die Berechnung würde sich auf ein lineares Problem reduzieren.
\end{enumerate}

